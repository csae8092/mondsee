
        \documentclass{article}
        \usepackage[ngerman]{babel}
        \usepackage[utf8]{inputenc}
       \usepackage{geometry}
       \geometry{
                 a4paper,
                 textwidth=150mm, %check below or in the xslt also the width of table rows
                 left=40mm,
                 top=40mm,
                 }
        \usepackage{natbib}
        \usepackage{graphicx}
        \renewcommand{\arraystretch}{1.7} %makes the space between rows larger
        \usepackage{hyperref}
        \usepackage{longtable}
        \usepackage{float}
        \restylefloat{table} % fix the position of tables
        \parskip=12pt
        
        \begin{document}
        \setcounter{secnumdepth}{0}
        \section{Cod. 1696 (Spiegelblatt)}
        \subsection{Pseudo-Hieronymus: Expositio quatuor evangeliorum}
        Pergament, 2 Einzelbl�tter, beschnitten \\
        9. Jh. Mondsee (?) 
             
\begin{longtable}[l]{@{}p{0,5cm}p{13cm}}
\textbf{M:} & Heutige Ma�e der Makulatur: VS: 190 x 150 mm; HS: 192 x 150 mm. Die stark verblasste Tinte deutet darauf hin, dass die Fragmente wohl gewaschen wurden; rostige L�cher von den Beschl�gen. Das dazugeh�rige Fragment in Cod. 1631 ist der obere Teil des hier auf dem Hinterspiegel aufgeklebten Fragmentes. \\
\textbf{B:} & Blattgr��e der urspr�nglichen Handschrift min 150 X 265 mm. \\
\textbf{S:} &
    Schriftraum: 115-120 X 153-155 mm. 1 Spalte zu 23 Zeilen (H�he: 8mm). Blindlinierung; links eine Spalte f�r Versalien. Karolingische Minuskel. Datierung nach dem pal�ographischen Befund: 9. Jh. \\

  \textbf{G:} &
    Nach der Makulierung diente das Fragment als Spiegel in Wien, �NB, Cod. 1696 (2.~H�lfte 13. Jh.). Inhalt des Tr�gerbandes: Sammelhandschrift. Provenienz Mondsee, Vorsignatur 'Lunael. q. 175'. Einband: Rotes Leder �ber Holzdeckeln mit Streicheisenlinien; eine Schlie�e und f�nf Buckeln auf je Deckel, entfernt. \\

  \textbf{Z:} &
    Fragmente derselben Handschrift: Berlin, Staatsbibl., Fragm. 47; Wien, �NB, Cod.~1361, Cod. 1696, Cod. Ser. n. 3754, Fragm. 782b. \\
\textbf{I:} &
    Pseudo-Hieronymus: Expositio quatuor evangeliorum
            \newline
            In evangelium secundum Matthaeum: (VS) [im]\textit{pleta in ioseph per dauid ...-... Ostenditur quod ipse uisionem domini meruit uidere }[qui fecit] (nach einigen Zeilen Textverlust, Fortsetzung
        auf dem HS) \textit{aptum fuit quia domus panis interpretatur ...-... non alia creatura sed muta} (Verso nur einige W�rter lesbar,
        Fortsetzung nach einigen Zeilen Textverlust) [ma]\textit{lis ostensio b}[ona] ...-... \textit{qua}[ndo
        magi]. \\
        \textbf{I:} &
    Pseudo-Hieronymus: Expositio quatuor evangeliorum
            \newline
            In evangelium secundum Matthaeum: (VS) [im]\textit{pleta in ioseph per dauid ...-... Ostenditur quod ipse uisionem domini meruit uidere }[qui fecit] (nach einigen Zeilen Textverlust, Fortsetzung
        auf dem HS) \textit{aptum fuit quia domus panis interpretatur ...-... non alia creatura sed muta} (Verso nur einige W�rter lesbar,
        Fortsetzung nach einigen Zeilen Textverlust) [ma]\textit{lis ostensio b}[ona] ...-... \textit{qua}[ndo
        magi]. \\
        \textbf{I:} &
    Pseudo-Hieronymus: Expositio quatuor evangeliorum
            \newline
            In evangelium secundum Matthaeum: (VS) [im]\textit{pleta in ioseph per dauid ...-... Ostenditur quod ipse uisionem domini meruit uidere }[qui fecit] (nach einigen Zeilen Textverlust, Fortsetzung
        auf dem HS) \textit{aptum fuit quia domus panis interpretatur ...-... non alia creatura sed muta} (Verso nur einige W�rter lesbar,
        Fortsetzung nach einigen Zeilen Textverlust) [ma]\textit{lis ostensio b}[ona] ...-... \textit{qua}[ndo
        magi]. \\
  \textbf{I:} &
    Pseudo-Hieronymus: Expositio quatuor evangeliorum
            \newline
            In evangelium secundum Matthaeum: (VS) [im]\textit{pleta in ioseph per dauid ...-... Ostenditur quod ipse uisionem domini meruit uidere }[qui fecit] (nach einigen Zeilen Textverlust, Fortsetzung
        auf dem HS) \textit{aptum fuit quia domus panis interpretatur ...-... non alia creatura sed muta} (Verso nur einige W�rter lesbar,
        Fortsetzung nach einigen Zeilen Textverlust) [ma]\textit{lis ostensio b}[ona] ...-... \textit{qua}[ndo
        magi]. \\

  \textbf{L:} &
    Bischoff, Katalog, S. 123 (ohne Erw�hnung dieses Fragments).
\end{longtable}

\begin{longtable}[l]{@{}p{0,5cm}p{13cm}}
\textbf{M:} & Heutige Ma�e der Makulatur: VS: 190 x 150 mm; HS: 192 x 150 mm. Die stark verblasste Tinte deutet darauf hin, dass die Fragmente wohl gewaschen wurden; rostige L�cher von den Beschl�gen. Das dazugeh�rige Fragment in Cod. 1631 ist der obere Teil des hier auf dem Hinterspiegel aufgeklebten Fragmentes. \\
\textbf{B:} & Blattgr��e der urspr�nglichen Handschrift min 150 X 265 mm. \\
\textbf{S:} &
    Schriftraum: 115-120 X 153-155 mm. 1 Spalte zu 23 Zeilen (H�he: 8mm). Blindlinierung; links eine Spalte f�r Versalien. Karolingische Minuskel. Datierung nach dem pal�ographischen Befund: 9. Jh. \\

  \textbf{G:} &
    Nach der Makulierung diente das Fragment als Spiegel in Wien, �NB, Cod. 1696 (2.~H�lfte 13. Jh.). Inhalt des Tr�gerbandes: Sammelhandschrift. Provenienz Mondsee, Vorsignatur 'Lunael. q. 175'. Einband: Rotes Leder �ber Holzdeckeln mit Streicheisenlinien; eine Schlie�e und f�nf Buckeln auf je Deckel, entfernt. \\

  \textbf{Z:} &
    Fragmente derselben Handschrift: Berlin, Staatsbibl., Fragm. 47; Wien, �NB, Cod.~1361, Cod. 1696, Cod. Ser. n. 3754, Fragm. 782b. \\
\textbf{I:} &
    Pseudo-Hieronymus: Expositio quatuor evangeliorum
            \newline
            In evangelium secundum Matthaeum: (VS) [im]\textit{pleta in ioseph per dauid ...-... Ostenditur quod ipse uisionem domini meruit uidere }[qui fecit] (nach einigen Zeilen Textverlust, Fortsetzung
        auf dem HS) \textit{aptum fuit quia domus panis interpretatur ...-... non alia creatura sed muta} (Verso nur einige W�rter lesbar,
        Fortsetzung nach einigen Zeilen Textverlust) [ma]\textit{lis ostensio b}[ona] ...-... \textit{qua}[ndo
        magi]. \\
        \textbf{I:} &
    Pseudo-Hieronymus: Expositio quatuor evangeliorum
            \newline
            In evangelium secundum Matthaeum: (VS) [im]\textit{pleta in ioseph per dauid ...-... Ostenditur quod ipse uisionem domini meruit uidere }[qui fecit] (nach einigen Zeilen Textverlust, Fortsetzung
        auf dem HS) \textit{aptum fuit quia domus panis interpretatur ...-... non alia creatura sed muta} (Verso nur einige W�rter lesbar,
        Fortsetzung nach einigen Zeilen Textverlust) [ma]\textit{lis ostensio b}[ona] ...-... \textit{qua}[ndo
        magi]. \\
        \textbf{I:} &
    Pseudo-Hieronymus: Expositio quatuor evangeliorum
            \newline
            In evangelium secundum Matthaeum: (VS) [im]\textit{pleta in ioseph per dauid ...-... Ostenditur quod ipse uisionem domini meruit uidere }[qui fecit] (nach einigen Zeilen Textverlust, Fortsetzung
        auf dem HS) \textit{aptum fuit quia domus panis interpretatur ...-... non alia creatura sed muta} (Verso nur einige W�rter lesbar,
        Fortsetzung nach einigen Zeilen Textverlust) [ma]\textit{lis ostensio b}[ona] ...-... \textit{qua}[ndo
        magi]. \\
  \textbf{I:} &
    Pseudo-Hieronymus: Expositio quatuor evangeliorum
            \newline
            In evangelium secundum Matthaeum: (VS) [im]\textit{pleta in ioseph per dauid ...-... Ostenditur quod ipse uisionem domini meruit uidere }[qui fecit] (nach einigen Zeilen Textverlust, Fortsetzung
        auf dem HS) \textit{aptum fuit quia domus panis interpretatur ...-... non alia creatura sed muta} (Verso nur einige W�rter lesbar,
        Fortsetzung nach einigen Zeilen Textverlust) [ma]\textit{lis ostensio b}[ona] ...-... \textit{qua}[ndo
        magi]. \\

  \textbf{L:} &
    Bischoff, Katalog, S. 123 (ohne Erw�hnung dieses Fragments).

\end{longtable}
        \end{document}